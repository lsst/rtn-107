% Copyright 2007 by Mark Wibrow
%
% This file may be distributed and/or modified
%
% 1. under the LaTeX Project Public License and/or
% 2. under the GNU Free Documentation License.
%
% See the file doc/generic/pgf/licenses/LICENSE for more details.

\section{Additional Mathematical Commands}

\label{pgfmath-commands}

Instead of parsing and evaluating complex expressions, you can also
use the mathematical engine to evaluate a single mathematical
operation. The macros used for many of these computations are listed
above in Section~\ref{pgfmath-functions}. \pgfname{} also provides
some additional commands which are shown below:

\subsection{Basic arithmetic functions}

\label{pgfmath-commands-basic}

In addition to the commands described in
Section~\ref{pgfmath-functions-basic}, the following command is
provided:

\begin{command}{\pgfmathreciprocal\marg{x}}
	Defines |\pgfmathresult| as $1\div\meta{x}$. This is provides
	greatest accuracy when \mvar{x} is small.
\end{command}

\subsection{Comparison and logical functions}

In addition to the commands described in
Section~\ref{pgfmath-functions-comparison},
the following command was provided by Christian Feuers\"anger:

\begin{command}{\pgfmathapproxequalto\marg{x}\marg{y}}
	Defines |\pgfmathresult| 1.0 if $ \rvert \meta{x} - \meta{y} \lvert < 0.0001$, but 0.0 otherwise.
	As a side-effect, the global boolean |\ifpgfmathcomparison| will be set accordingly.
\end{command}

\subsection{Pseudo-Random Numbers}

\label{pgfmath-random}

In addition to the commands described in
Section~\ref{pgfmath-functions-random},
the following commands are provided:

\begin{command}{\pgfmathgeneratepseudorandomnumber}
	Defines |\pgfmathresult| as a pseudo-random integer between 1 and
	$2^{31}-1$. This uses a linear congruency generator, based on ideas
	due to Erich Janka.
\end{command}

\begin{command}{\pgfmathrandominteger\marg{macro}\marg{maximum}\marg{minimum}}
	This defines \meta{macro} as a pseudo-randomly generated integer from
	the range \meta{maximum} to \meta{minimum} (inclusive).

\begin{codeexample}[]
\begin{pgfpicture}
   \foreach \x in {1,...,50}{
      \pgfmathrandominteger{\a}{1}{50}
      \pgfmathrandominteger{\b}{1}{50}
      \pgfpathcircle{\pgfpoint{+\a pt}{+\b pt}}{+2pt}
      \color{blue!40!white}
      \pgfsetstrokecolor{blue!80!black}
      \pgfusepath{stroke, fill}
   }
\end{pgfpicture}
\end{codeexample}
\end{command}

\begin{command}{\pgfmathdeclarerandomlist\marg{list name}\{\marg{item-1}\marg{item 2}...\}}
	This creates a list of items with the name \meta{list name}.
\end{command}

\begin{command}{\pgfmathrandomitem\marg{macro}\marg{list name}}
	Select an item from a random list \meta{list name}. The
	selected item is placed in \meta{macro}.
\end{command}

\begin{codeexample}[]
\begin{pgfpicture}
   \pgfmathdeclarerandomlist{color}{{red}{blue}{green}{yellow}{white}}
   \foreach \a in {1,...,50}{
      \pgfmathrandominteger{\x}{1}{85}
      \pgfmathrandominteger{\y}{1}{85}
      \pgfmathrandominteger{\r}{5}{10}
      \pgfmathrandomitem{\c}{color}
      \pgfpathcircle{\pgfpoint{+\x pt}{+\y pt}}{+\r pt}
      \color{\c!40!white}
      \pgfsetstrokecolor{\c!80!black}
      \pgfusepath{stroke, fill}
   }
\end{pgfpicture}
\end{codeexample}

\begin{command}{\pgfmathsetseed\marg{integer}}
  Explicitly set seed for the pseudo-random number generator. By
  default it is set to the value of |\time|$\times$|\year|.
\end{command}



\subsection{Base Conversion}

\label{pgfmath-bases}

\pgfname{} provides limited support for conversion between
\emph{representations} of numbers. Currently the numbers must be
positive integers in the range $0$ to $2^{31}-1$, and the bases in the
range $2$ to $36$. All digits representing numbers greater than 9 (in
base ten), are alphabetic, but may be upper or lower case.

In addition to the commands described in
Section~\ref{pgfmath-functions-base},
the following commands are provided:

\begin{command}{\pgfmathbasetodec\marg{macro}\marg{number}\marg{base}}
	Defines \meta{macro} as the result of converting \meta{number} from
	base \meta{base} to base 10. Alphabetic digits can be upper or lower
	case.

\medskip{\def\medskip{}

\begin{codeexample}[]
\pgfmathbasetodec\mynumber{107f}{16} \mynumber
\end{codeexample}


\begin{codeexample}[]
\pgfmathbasetodec\mynumber{33FC}{20} \mynumber
\end{codeexample}

}\medskip

\end{command}

\begin{command}{\pgfmathdectobase\marg{macro}\marg{number}\marg{base}}
	Defines \meta{macro} as the result of converting \meta{number} from
	base 10 to base \meta{base}. Any resulting alphabetic digits are in
	\emph{lower case}.

\begin{codeexample}[]
\pgfmathdectobase\mynumber{65535}{16} \mynumber
\end{codeexample}

\end{command}

\begin{command}{\pgfmathdectoBase\marg{macro}\marg{number}\marg{base}}
	Defines \meta{macro} as the result of converting \meta{number} from
	base 10 to base \meta{base}. Any resulting alphabetic digits are in
	\emph{upper case}.

\begin{codeexample}[]
\pgfmathdectoBase\mynumber{65535}{16} \mynumber
\end{codeexample}

\end{command}

\begin{command}{\pgfmathbasetobase\marg{macro}\marg{number}\marg{base-1}\marg{base-2}}
	Defines \meta{macro} as the result of converting \meta{number} from
	base \meta{base-1} to base \meta{base-2}. Alphabetic digits in
	\meta{number} can be upper or lower case, but any resulting
	alphabetic digits are in \emph{lower case}.

\begin{codeexample}[]
\pgfmathbasetobase\mynumber{11011011}{2}{16} \mynumber
\end{codeexample}

\end{command}

\begin{command}{\pgfmathbasetoBase\marg{macro}\marg{number}\marg{base-1}\marg{base-2}}
	Defines \meta{macro} as the result of converting \meta{number} from
	base \meta{base-1} to base \meta{base-2}. Alphabetic digits in
	\meta{number} can be upper or lower case, but any resulting
	alphabetic digits are in \emph{upper case}.

\begin{codeexample}[]
\pgfmathbasetoBase\mynumber{121212}{3}{12} \mynumber
\end{codeexample}

\end{command}


\begin{command}{\pgfmathsetbasenumberlength\marg{integer}}
	Set the number of digits in the result of a base conversion to
	\meta{integer}. If the result of a conversion has less digits
	than this number it is prefixed with zeros.

\begin{codeexample}[]
\pgfmathsetbasenumberlength{8}
\pgfmathdectobase\mynumber{15}{2} \mynumber
\end{codeexample}

\end{command}

