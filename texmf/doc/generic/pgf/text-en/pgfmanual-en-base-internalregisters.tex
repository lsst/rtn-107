% Copyright 2008 by Christian Feuersaenger
%
% This file may be distributed and/or modified
%
% 1. under the LaTeX Project Public License and/or
% 2. under the GNU Free Documentation License.
%
% See the file doc/generic/pgf/licenses/LICENSE for more details.


\section{Adding libraries to \pgfname: temporary registers}

\label{section-internal-registers}

This section is intended for those who like to write libraries to extend \pgfname. Of course, this requires a good deal of knowledge about \TeX-programming and the structure of the \pgfname\ basic layer. Besides, one will encounter the need of temporary variables and, especially, temporary \TeX\ registers. This section describes how to use a set of pre-allocated temporary registers of the basic layer without needing to allocate more of them.

A part of these internals are already mentioned in section~\ref{section-internal-pointcmds}, but the basic layer provides more temporaries than |\pgf@x| and |\pgf@y|.

\begin{internallist}[dimen register]{\pgf@x,\pgf@y}
	These registers are used to process point coordinates in the basic layer of \pgfname, see section~\ref{section-internal-pointcmds}. After a |\pgfpoint|$\dotsc$ command, they contain the final $x$ and $y$ coordinate, respectively.

	The values of |\pgf@x| and |\pgf@y| are set \emph{globally} in contrast to other available \pgfname\ registers. You should never assume anything about their value unless the context defines them explicitly.


	Please prefer the |\pgf@xa|, |\pgf@xb|, $\dotsc$ registers for temporary dimen registers unless you are writing point coordinate commands.
\end{internallist}

\begin{internallist}[dimen register]{
	\pgf@xa,
	\pgf@xb,
	\pgf@xc,
	\pgf@ya,
	\pgf@yb,
	\pgf@yc}
	Temporary registers for \TeX\ dimensions which can be modified freely. Just make sure changes occur only within \TeX\ groups.

	\paragraph{Attention:}
	\pgfname\ uses these registers to perform path operations. For reasons of efficiency, path commands do not always guard them. As a consequence, the code
\begin{codeexample}[code only]
\pgfpointadd{\pgfpoint{\pgf@xa}{\pgf@ya}}{\pgfpoint{\pgf@xb}{\pgf@yb}}
\end{codeexample}
\noindent
may fail: Inside |\pgfpointadd|, the |\pgf@xa| and friend registers might be
modified. In particular, it might happen that |\pgf@xb| is changed
before |\pgfpoint{\pgf@xb}{\pgf@yb}| is evaluated. The right thing to do
would be to first expand everything using |\edef| and process the values afterwards,
resulting in unnecessary expensive operations. Of course, one can avoid this by simply
looking into the source code of |\pgfpointadd| to see which registers are used.
\end{internallist}

\begin{internallist}[dimen register]{\pgfutil@tempdima,\pgfutil@tempdimb}
	Further multi-purpose temporary dimen registers. For \LaTeX, these registers are already allocated as |\@tempdima| and |\@tempdimb| and are simply |\let| to the |\pgfutil@|$\dotsc$ names.
\end{internallist}

\begin{internallist}[count register]{
	\c@pgf@counta,
	\c@pgf@countb,
	\c@pgf@countc,
	\c@pgf@countd}
	These multiple-purpose count registers are used throughout \pgfname\ to perform integer computations. Feel free to use them as well, just make sure changes are scoped by local \TeX\ groups.
\end{internallist}

\begin{internallist}[openout handle]{\w@pgf@writea}
	An |\openout| handle which is used to generate complete output files within locally scoped parts of \pgfname\ (for example, to interact with |gnuplot|). You should always use |\immediate| in front of output operations involving |\w@pgf@writea| and you should always close the file before returning from your code.
\begin{codeexample}[code only]
\immediate\openout\w@pgf@writea=myfile.dat
\immediate\write\w@pgf@writea{...}%
\immediate\write\w@pgf@writea{...}%
\immediate\closeout\w@pgf@writea%
\end{codeexample}
\end{internallist}

\begin{internallist}[openin handle]{\r@pgf@reada}
	An |\openin| handle which is used to read files within locally scoped parts of \pgfname, for example to check if a file exists or to read data files. You should always use |\immediate| in front of output operations involving |\w@pgf@writea| and you should always close the file before returning from your code.
\begin{codeexample}[code only]
\immediate\openin\r@pgf@reada=myfile.dat
% do something with \macro
\ifeof\r@pgf@reada
    % end of file or it doesn't exist
\else
    % loop or whatever
    \immediate\read\r@pgf@reada to\macro
    ...
\fi
\immediate\closein\r@pgf@reada
\end{codeexample}
\end{internallist}

\begin{internallist}[box]{\pgfutil@tempboxa}
	A box for temporary use inside of local \TeX\ scopes. For \LaTeX, this box is the same as the already pre-allocated |\@tempboxa|.
\end{internallist}

