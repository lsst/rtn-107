\documentclass[OPS,lsstdraft,authoryear,toc]{lsstdoc}
% GENERATED FILE -- edit this in the Makefile
\newcommand{\lsstDocType}{RTN}
\newcommand{\lsstDocNum}{107}
\newcommand{\vcsRevision}{491e809-dirty}
\newcommand{\vcsDate}{2025-09-05}


% Package imports go here.

% Local commands go here.

%If you want glossaries
%\input{aglossary.tex}
%\makeglossaries

\title{The Rubin Observatory Target-of-Opportunity Mock Data Challenge}

% This can write metadata into the PDF.
% Update keywords and author information as necessary.
\hypersetup{
    pdftitle={The Rubin Observatory Target-of-Opportunity Mock Data Challenge},
    pdfauthor={Sean PatrickMacBride},
    pdfkeywords={}
}

% Optional subtitle
% \setDocSubtitle{A subtitle}

\input{authors}

\setDocRef{RTN-107}
\setDocUpstreamLocation{\url{https://github.com/lsst/rtn-107}}

\date{\vcsDate}

% Optional: name of the document's curator
% \setDocCurator{The Curator of this Document}

\setDocAbstract{%
We describe the activities of the Target-of-Opportunity mock data challenge
}

% Change history defined here.
% Order: oldest first.
% Fields: VERSION, DATE, DESCRIPTION, OWNER NAME.
% See LPM-51 for version number policy.
\setDocChangeRecord{%
  \addtohist{1}{YYYY-MM-DD}{Unreleased.}{MacBride}
}


\begin{document}

% Create the title page.
\maketitle
% Frequently for a technote we do not want a title page  uncomment this to remove the title page and changelog.
% use \mkshorttitle to remove the extra pages

% ADD CONTENT HERE
% You can also use the \input command to include several content files.

\appendix

\section{Acknowledgements}

This material is based upon work supported in part by the National Science Foundation through Cooperative Agreements AST-1258333 and AST-2241526 and Cooperative Support Agreements AST-1202910 and AST-2211468 managed by the Association of Universities for Research in Astronomy (AURA), and the Department of Energy under Contract No.\ DE-AC02-76SF00515 with the SLAC National Accelerator Laboratory managed by Stanford University.
Additional Rubin Observatory funding comes from private donations, grants to universities, and in-kind support from LSST-DA Institutional Members.

% Include all the relevant bib files.
% https://lsst-texmf.lsst.io/lsstdoc.html#bibliographies
\section{References} \label{sec:bib}
\renewcommand{\refname}{} % Suppress default Bibliography section
\bibliography{local,lsst,lsst-dm,refs_ads,refs,books}

% Make sure lsst-texmf/bin/generateAcronyms.py is in your path
\section{Acronyms} \label{sec:acronyms}
\addtocounter{table}{-1}
\begin{longtable}{p{0.145\textwidth}p{0.8\textwidth}}\hline
\textbf{Acronym} & \textbf{Description}  \\\hline

AST & NSF Division of Astronomical Sciences \\\hline
AURA & Association of Universities for Research in Astronomy \\\hline
DE-AC02 & Department of Energy contract number prefix \\\hline
FOV & field of view \\\hline
GW & Gravitational Wave \\\hline
LSST & Legacy Survey of Space and Time (formerly Large Synoptic Survey Telescope) \\\hline
LSST-DA & LSST Discovery Alliance \\\hline
LVK & LIGO-Virgo-KAGRA \\\hline
PSTN & Project Science Technical Note \\\hline
RTN & Rubin Technical Note \\\hline
SLAC & SLAC National Accelerator Laboratory \\\hline
SV & Science Validation \\\hline
ToO & Target of Opportunity \\\hline
\end{longtable}

% If you want glossary uncomment below -- comment out the two lines above
%\printglossaries





\end{document}
