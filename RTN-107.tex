\documentclass[OPS,lsstdraft,authoryear,toc]{lsstdoc}
% GENERATED FILE -- edit this in the Makefile
\newcommand{\lsstDocType}{RTN}
\newcommand{\lsstDocNum}{107}
\newcommand{\vcsRevision}{491e809-dirty}
\newcommand{\vcsDate}{2025-09-05}


% Package imports go here.

% Local commands go here.

%If you want glossaries
%\input{aglossary.tex}
%\makeglossaries

\title{The Rubin Observatory Target-of-Opportunity Mock Data Challenge}

% This can write metadata into the PDF.
% Update keywords and author information as necessary.
\hypersetup{
    pdftitle={The Rubin Observatory Target-of-Opportunity Mock Data Challenge},
    pdfauthor={Sean Patrick MacBride},
    pdfkeywords={}
}

% Optional subtitle
% \setDocSubtitle{A subtitle}

\input{authors}

\setDocRef{RTN-107}
\setDocUpstreamLocation{\url{https://github.com/lsst/rtn-107}}

\date{\vcsDate}

% Optional: name of the document's curator
% \setDocCurator{The Curator of this Document}

\setDocAbstract{%
We describe the activities of the Target-of-Opportunity mock data challenge, taking place from Sep 22 2025 - Oct 18 2025. We center this activity in four questions that are critical for maximizing the scientific output of the ToO system:
\begin{itemize}
    \item How quickly can Rubin Observatory start observing after a ToO alert is received?
    \item How efficient is Rubin Observatory at recovering the host of a ToO event?
    \item How accurate are the observing strategies that the community has created for the ToO program?
    \item How can expert ToO scientists interact effectively with the ToO system, where many processes are fully automated?
\end{itemize}
In this challenge, and the report summarized herein, we aim to answer the aforementioned questions to better the Rubin ToO program.
}

% Change history defined here.
% Order: oldest first.
% Fields: VERSION, DATE, DESCRIPTION, OWNER NAME.
% See LPM-51 for version number policy.
\setDocChangeRecord{%
  \addtohist{1}{2025-09-18}{Introduction + Abstract.}{MacBride}
}


\begin{document}

% Create the title page.
\maketitle
% Frequently for a technote we do not want a title page  uncomment this to remove the title page and changelog.
% use \mkshorttitle to remove the extra pages

% ADD CONTENT HERE
% You can also use the \input command to include several content files.

\section{Introduction}

Rubin Observatory will enable many scientific discoveries, including target-of-opportunity (ToO) observations of GW, high energy neutrino, potentially hazardous asteroids, and galactic supernova. 

The $\sim10\deg^2$ field-of-view of the Rubin optical system allow ToO observations to survey a wide area, while the single-visit depth of the LSST-Camera allows single observations to rapidly observe the southern sky for faint transient phenomena. The combination of the large FOV and deep observations make Rubin Observatory an ideal tool for discovery of ToO phenomena.

The Rubin ToO program encompasses 3\% of the LSST. Each target has a different observing strategy based on observing conditions, the conditions of the astrophysical event, and other parameters. The observing strategies are the product of community input, and were revised in 2024 (\cite{RubinToO2024}). These recommendations were accepted by the survey cadence optimization committee in January 2025 \citep{PSTN-056}.

The Rubin ToO mock data challenge was spawned out of necessity to properly characterize the infrastructure and scientific potential of 3\% of the LSST. This challenge has included coordination with the LVK collaboration and the IceCube collaboration, who will be sending simulated data that is syntactically identical to real alerts. The action on the Rubin Observatory side is to parse and process these alerts as if we were in operations to obtain a better understanding of the ToO system.

The deliverables of the mock data challenge are grounded in characterizing the technical and scientific outcomes of the ToO program. They are the following:

\begin{itemize}
    \item \textbf{How quickly can Rubin Observatory start observing after a ToO alert is received?} While there will be variance in ToO response time due to slew time being non-uniformly distributed, many aspects of the ToO time-to-response can be adequately measured with many additional ToO alerts.
    \item \textbf{How efficient is Rubin Observatory at recovering the host of a ToO event?} Previous forecasts of candidate discoverability have been grounded in simulations, while this exercise utilizes data from the on-sky performance of Rubin observatory during the SV survey.
    \item \textbf{Are changes to the community designed observing strategies necessary?} To adequately serve the Rubin ToO community, the observing strategies must reflect the desired outcomes of the multi-messenger community as requested.
    \item \textbf{How can expert ToO scientists interact effectively with the ToO system, where many processes are fully automated?} This includes pursuing ToO's that were outside of the original alert criteria, stopping a ToO observation after it has begun, and gathering sufficient information to assess a specific ToO alert.
\end{itemize}

We describe the events and results of the first Rubin Observatory Target-of-Opportunity mock data challenge. In section \ref{sec:organization}, we describe the involvement of different experiments, and describe how they interact with the Rubin ToO ecosystem. In section \ref{sec:challenge}, we describe the events and activities of the data challenge itself. In section \ref{sec:latency}, we describe analyses of alert latency. In section \ref{sec:efficiency}, we describe analyses of observing efficiency. In section \ref{sec:stratGen}, we describe analyses of observing strategy generation. In section \ref{sec:human}, we describe the interaction between scientists and ToO infrastructure. Finally, in section \ref{sec:conclusion}, we conclude on the results herein and reflect ahead of future ToO activities.

\newpage
\section{Organization}\label{sec:organization}

Here, describe LVK involvement, IceCube involvement, SCiMMA, Rubin infrastructure.

\newpage
\section{The challenge}\label{sec:challenge}

Here, we describe the events of the MDC

\newpage
\section{Alert latency}\label{sec:latency}

Here, we describe the analyses of the alert latency

\subsection{LVK Alerts}\label{subsec:latency-LVK}

\subsection{IceCube Alerts}\label{subsec:latency-IceCube}


\newpage
\section{Observing efficiency}\label{sec:efficiency}

Here, we describe the analyses on the observing efficiency and candidate discoverability.

\subsection{LVK Alerts}\label{subsec:Efficiency-LVK}

\subsection{IceCube Alerts}\label{subsec:Efficiency-IceCube}

\newpage
\section{Strategy generation}\label{sec:stratGen}

Here, we describe the analyses of observing strategy generation

\subsection{LVK Alerts}\label{subsec:strategy-LVK}

\subsection{IceCube Alerts}\label{subsec:strategy-IceCube}

\newpage
\section{The Human Component}\label{sec:human}

Here, we describe the human component of this activity

\newpage
\section{Conclusion}\label{sec:conclusion}

Here, we review the collected results, and assess the status and readiness of the ToO program.

\newpage
\appendix

\section{Acknowledgements}

This material is based upon work supported in part by the National Science Foundation through Cooperative Agreements AST-1258333 and AST-2241526 and Cooperative Support Agreements AST-1202910 and AST-2211468 managed by the Association of Universities for Research in Astronomy (AURA), and the Department of Energy under Contract No.\ DE-AC02-76SF00515 with the SLAC National Accelerator Laboratory managed by Stanford University.
Additional Rubin Observatory funding comes from private donations, grants to universities, and in-kind support from LSST-DA Institutional Members.

% Include all the relevant bib files.
% https://lsst-texmf.lsst.io/lsstdoc.html#bibliographies
\section{References} \label{sec:bib}
\renewcommand{\refname}{} % Suppress default Bibliography section
\bibliography{local,lsst,lsst-dm,refs_ads,refs,books}

% Make sure lsst-texmf/bin/generateAcronyms.py is in your path
\section{Acronyms} \label{sec:acronyms}
\addtocounter{table}{-1}
\begin{longtable}{p{0.145\textwidth}p{0.8\textwidth}}\hline
\textbf{Acronym} & \textbf{Description}  \\\hline

AST & NSF Division of Astronomical Sciences \\\hline
AURA & Association of Universities for Research in Astronomy \\\hline
DE-AC02 & Department of Energy contract number prefix \\\hline
FOV & field of view \\\hline
GW & Gravitational Wave \\\hline
LSST & Legacy Survey of Space and Time (formerly Large Synoptic Survey Telescope) \\\hline
LSST-DA & LSST Discovery Alliance \\\hline
LVK & LIGO-Virgo-KAGRA \\\hline
PSTN & Project Science Technical Note \\\hline
RTN & Rubin Technical Note \\\hline
SLAC & SLAC National Accelerator Laboratory \\\hline
SV & Science Validation \\\hline
ToO & Target of Opportunity \\\hline
\end{longtable}

% If you want glossary uncomment below -- comment out the two lines above
%\printglossaries





\end{document}
